\chapter{パラメータ}
\label{chap:parameter}

パラメータには整数パラメータ(param\textunderscore{}int)と、文字列パラメータ(param\textunderscore{}str)があります。NSLでは、パラメータは二つの使い方をします。一つ目は、構造展開を行う時の制御変数としての使い方です。制御変数として用いる場合には、パラメータに値を設定します。この場合、パラメータは、モジュールの外に定義し、大域変数扱いになります。

\begin{reviewemlist}
\begin{alltt}
param\textunderscore{}int パラメータ名 = 整数;

param\textunderscore{}str パラメータ名 = 文字列;
\end{alltt}
\end{reviewemlist}

if文の条件式において、パラメータ名を用いて比較演算を行うことができます。整数パラメータの比較は、整数変数と同一の演算子を用います。文字列の比較には、一致(==)と不一致(!=)が利用可能です。

\begin{reviewemlist}
\begin{alltt}
param\textunderscore{}str MODE = "SIM";

module cpu \{
 if(MODE == "SIM") \{
   \textunderscore{}display("PC: \%4X", pc);
 \}
\end{alltt}
\end{reviewemlist}

他の使い方として、NSL以外の言語(VerilogHDLやVHDL)で作成されたモジュールへのパラメータ設定があります。他言語のモジュールをインスタンスとして利用する場合に、場合によって、それらのモジュールに指定されたパラメータを与える必要があります。このため、他言語のモジュールのプロトタイプを指定するdeclare文において、パラメータを宣言することができます。

\begin{reviewemlist}
\begin{alltt}
param\textunderscore{}int パラメータ名;
param\textunderscore{}str パラメータ名;
\end{alltt}
\end{reviewemlist}

サブモジュールのインスタンス宣言において、パラメータに値をセットすることで、生成されたVerilogHDLもしくはVHDLにパラメータを渡します。

\begin{reviewemlist}
\begin{alltt}
モジュール名 サブモジュールインスタンス名(パラメータ名=値, パラメータ名=値) ;
\end{alltt}
\end{reviewemlist}

初期値を設定した整数パラメータと、文字列パラメータは、整数もしくは文字列として、if文の構造展開の条件式に用いることができます。

次の記述は、Xilinx社のFPGAのクロックモジュール(DCM)とグローバルバッファ(BUFG)を用いて、クロック配線を作る例題です。DCMには、ユーザが与えなくてはならないパラメータがあり、これをNSLから設定しています。この例題は、論理合成専用の例題で、一部のモジュールはdeclareしかないしシミュレーション実行部分がないので、実行して確かめることはできませんが、記述方法の参考にしてください。

\reviewlistcaption{リスト14.1: tut12}
\begin{reviewlist}
\begin{alltt}
declare DCM interface \{
param\textunderscore{}int CLKDV\textunderscore{}DIVIDE;
param\textunderscore{}str CLK\textunderscore{}FEEDBACK;
input RST, PSINCDEC, PSEN, PSCLK, CLKIN, CLKFB;
output PSDONE, CLK0, CLK90, CLK180, CLK270,
 CLK2X, CLK2X180, CLKDV, CLKFX, CLKFX180,
 LOCKED, STATUS[8];
\}
declare BUFG interface \{
input I;
output O;
\}

declare sample \{
input samplein;
output sampleout;
\}

declare tut12 \{\}
module tut12 \{
BUFG buff;
DCM dcm2(CLKDV\textunderscore{}DIVIDE=4, CLK\textunderscore{}FEEDBACK="1X");
sample target;
dcm2.RST = p\textunderscore{}reset;
dcm2.CLKIN = m\textunderscore{}clock;
dcm2.CLKFB = dcm2.CLK0;
dcm2.PSEN = 0;
dcm2.PSCLK = 0;
dcm2.PSINCDEC =0;
buff.I = dcm2.CLKDV;
target.m\textunderscore{}clock = buff.O;
target.p\textunderscore{}reset = p\textunderscore{}reset;
\}
\end{alltt}
\end{reviewlist}


\chapter{はじめに}
\label{chap:preface}

NSLは論理設計のために開発された言語です。
どんな言語を習う場合にも、
動作するサンプルコードを実際に動かしながら学ぶ
方法がもっとも効果的でしょう。

NSLによって、多くの人々が論理設計の楽しさに目覚めることを
希望しています。

\section*{内容について}

このチュートリアルでは、NSLの基本的な機能から
少し高度な内容まで、動作するサンプルコードを
実際に動かしながら学んでいきます。
読者は、手元に処理系を用意して、チュートリアルに
出てくるサンプルを実際に動かしながら、その動作を
理解して欲しいと思います。

\section*{動作環境について}

本チュートリアルの動作は、\href{http://www.ip-arch.jp/}{IP ARCH}から
ダウンロード可能なLiveCygwin上で動作を確認しています。
NSLCORE 20111120版での確認を行っていますが、
これ以降のバージョンでも問題なく動作すると思います。

チュートリアルの内容をすべて網羅するには、

\begin{itemize}
\item NSLCORE
\item IcarusVerilog
\item GTKWave
\item SystemC
\end{itemize}

の各プログラムと、その動作環境が必要になります。

LiveCygwinには必要な環境がすべて用意されていますので、
これをダウンロードして展開することで、
個々のプログラムを準備する手間が大幅に省けます。

\section*{謝辞}

(必要に応じて)

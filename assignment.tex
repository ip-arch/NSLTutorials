\chapter{転送と演算、および複合文}
\label{chap:assignment}

NSLの構成要素である、レジスタと端子には、値を転送できます。転送を受けると、端子やレジスタは、ビット幅を持った値を持ちます。レジスタはクロックをまたいで値を保持するのに対し、端子は、転送が起きたクロックの間だけ有効な値を持ちます。そこで、この二つの転送を区別するために、NSLでは、レジスタへの転送記号(:=)と端子への転送記号(=)を異なる記号としています。また、値を記憶する構成要素であるメモリは、レジスタと同じ記号を持ちます。

レジスタへの転送は、転送を行ったクロックサイクルの次のクロックサイクルのはじめに値の変化が起きます。つまり、レジスタはリーディングエッジトリガフリップフロップで実現します。

レジスタや端子の値を用いて、演算を行うことができます。NSLには算術演算と論理演算があります。NSLの演算は、すべて、同一クロックにおいて結果を生じます。

算術演算は、次の3つを用意しています。除算は、実現方法が多数あり、かつ、1クロックで値を返す実装を用いることはほとんどないため、演算子として用意していません。

\begin{itemize}
\item + : 同一ビット幅同士の加算を行います。結果は同じビット幅の値となります。第2項に整数を指定すると第1項のビット幅に推定します。
\item - :  同一ビット幅同士の減算を行います。結果は同じビット幅の値となります。第2項に整数を指定すると第1項のビット幅に推定します。
\item * : 乗算を行います。ビット幅の制限はありません。結果は、第1項と第2項の和のビット幅を持ちます。
\end{itemize}

論理演算には、対応するビットごとに演算を行う演算と、1語のビット方向に演算を行うリデュース演算があります。リデュース演算は、値の前に前置演算子を置き、演算結果は1ビットとなります。ビットごとの演算は、同一ビット幅同士の演算を行い、結果のビット幅は変わりません。ビットごとの演算では、ビット幅の推定が可能なので、第2項には整数が使えます。次の演算の中で否定以外の演算はリデュース演算子にも利用できます。リデュース演算は1ビットの信号には適用できません。

\begin{itemize}
\item \&:論理積
\item \textbar{}:論理和
\item \textasciicircum{}:排他的論理和
\item \textasciitilde{}:否定
\end{itemize}

論理演算の例を示します。

\begin{reviewemlist}
\begin{alltt}
   a = 8'h55 \& 170;
   b = \textbar{}a;
\end{alltt}
\end{reviewemlist}

論理演算と比較演算の違いに十分注意してください。リデュース演算を用いた複雑な式を書く場合には、混乱しないよう、適切に括弧を用いてください。

その他の演算として、シフト、ビット連結、ビット切り出し、リピート、ビット幅変更、サイン拡張、条件演算があります。

\begin{itemize}
\item \textgreater{}\textgreater{}        論理右シフト。結果のビット幅は変わりません。第2項は整数にしてください。
\item \textless{}\textless{}        論理左シフト。結果のビット幅は変わりません。第2項は整数にしてください。
\item \{ 式 , 式, ... \}        \{ \} で囲む複数の式を、並び順にビット連結た値を返します。
\item 数値 \{ 式 \}        式を数値回数リピートして連結した値を返します。
\item 式[数値1:数値2]   式を数値1ビット目から数値2ビット目まで切り出した値を返します。
\item 式[式x]        	式から式xビットの1ビットを切り出した値を返します。
\item 数値 '( 式 )       式のビット幅を数値で示す幅に変更した値を返します。
\item 数値\#(式)        	式を数値ビットにサイン拡張した値を返します。
\item if ( 式 ) 式1 else 式2        式が真のとき、式1を、偽のときに式2を値として返します。
\end{itemize}

NSLは、演算のビット数を厳しくチェックします。そのため、設計者は、式の途中で、任意の式もしくは数値に対して、ビット数を変更したい場合が生じるでしょう。その場合には、次のようにビット数の変更を行ってください。(ここで、yは8ビットの端子とします)

\begin{reviewemlist}
\begin{alltt}
       y = 8'(12);
\end{alltt}
\end{reviewemlist}

演算式の一部だけを条件によって変更した場合、if演算を用います。if演算は、式1もしくは式2を値とします。たとえば、端子xにaとbのうち小さいほうを転送する場合、次のように記述します。

\begin{reviewemlist}
\begin{alltt}
       x = if(a\textless{}b) a else b;
\end{alltt}
\end{reviewemlist}

このほかに、記憶素子であるレジスタに対しては、++と--の演算子があります。++は、レジスタの値を1増やし、--は、レジスタ の値を1減らします。これらの演算子を、レジスタに限定しているのは、レジスタは、値が反映されるタイミングが次のクロックの立ち上がりであるため、レジス タの値を用いた結果を、そのレジスタに転送しても正しく演算できますが、端子の値を用いた演算結果を、同じ端子に転送すると、組合せ回路だけからなるループ回路 が生成されてしまい、回路動作が正しく行われないからです。この演算子は、式として使うと、前置と後置の二通りの使い方ができます。

\begin{reviewemlist}
\begin{alltt}
       w = r++;
\end{alltt}
\end{reviewemlist}

後置の場合には、もとのレジスタの値を式の値とし、その後、レジスタを加算もしくは減算します。

この例では、端子wにレジスタrの値を転送します。

\begin{reviewemlist}
\begin{alltt}
       y = --q;
\end{alltt}
\end{reviewemlist}

前置の場合には、加算もしくは減算した結果を式の値として、その結果をレジスタに転送します。 この例では、端子yにレジスタq-1の値を転送します。

また、演算を、動作として単独で記述することもできます。

\begin{reviewemlist}
\begin{alltt}
       s--;
\end{alltt}
\end{reviewemlist}

この場合、レジスタsの値を1だけ減算します。

中括弧 \{ \} を用いて、複数の文からなる複合文を作ることができます。複合文には次の形式があります。

\begin{itemize}
\item \{ \} : 複合文中のすべての文を同時に実行します。
\end{itemize}

\begin{reviewemlist}
\begin{alltt}
if(x==0) \{
   y=1;
   z=2;
\}
\end{alltt}
\end{reviewemlist}

xが0のとき、端子yに1を、端子zに2をそれぞれ同時に転送する。

\begin{itemize}
\item seq \{ \} : 1クロックずつ順次実行するシーケンスブロックを記述します。関数の処理としてのみ記述可能です。それぞれの文はパイプラインを構成します。すなわち、実行された文が終了する前に、さらに次の起動をした場合、後続の処理は先行する処理の終了を待たずに実行を開始します。
\end{itemize}

\begin{reviewemlist}
\begin{alltt}
func foo seq \{
   a=1;
   a=2;
   a=3;
\}
\end{alltt}
\end{reviewemlist}

fooが呼ばれると、端子aに1,2,3を順次転送します。

次の例題をtut2.nslという名前のテキストファイルとして用意してください。

\reviewlistcaption{リスト5.1: tut2}
\begin{reviewlist}
\begin{alltt}
declare tut2 simulation \{ \}

module tut2 \{
   reg count[5]=0;
   wire x[5],y[5];
   any \{
       count \textless{} 10 : \{
               \textunderscore{}display("x=\%d,y=\%d,count=\%d",x,y,count);
               count := x;
               y = x + 1;
               x = count + 1;
               \}
       count \textgreater{}= 10:    \textunderscore{}finish("End");
   \}
\}
\end{alltt}
\end{reviewlist}


例題tut2は少し意地悪な順序で記述を並べています。先を読む前にどのような結果が出てくるか考えてみてください。

ヒントは、複合文の各文は、すべて同時に実行されるということです。文の記述順序は実行状況に関係しません。

シミュレーションを実行するために、この回路を、NSL COREでコンパイルします。

\begin{reviewcmd}
\begin{alltt}
\textgreater{} nsl2vl tut2.nsl -verisim2 -target tut2
\end{alltt}
\end{reviewcmd}

この操作で、tut2.vというVerilogHDLのファイルができます。

次に、Icarus Verilogで、コンパイルします。

\begin{reviewcmd}
\begin{alltt}
\textgreater{} iverilog -o tut2.vvp tut2.v
\end{alltt}
\end{reviewcmd}

それでは、実行してみましょう。

\begin{reviewcmd}
\begin{alltt}
\textgreater{} vvp tut2.vvp

VCD info: dumpfile tut2.vcd opened for output.
x= 1,y= 2,count= 0
x= 1,y= 2,count= 0
x= 2,y= 3,count= 1
x= 3,y= 4,count= 2
x= 4,y= 5,count= 3
x= 5,y= 6,count= 4
x= 6,y= 7,count= 5
x= 7,y= 8,count= 6
x= 8,y= 9,count= 7
x= 9,y=10,count= 8
x=10,y=11,count= 9
End
\end{alltt}
\end{reviewcmd}

複合文の中はすべて同時に実行し、記述順には関係しないので、xとyへの転送と、countへの転送が同時に起こります。しかし、レジスタの値が変化するのは、次のクロックの立ち上がり時点なので、xよりも1クロック遅く変化が発生します。countが0の行が2行出ているのは、この回路には、リセット信号が入っていて、リセットがかかっている間、countが0のままとなるからです。

左辺ビット連結演算 .\{\}によって複数の端子やレジスタに、まとめて値を転送することができます。

\begin{itemize}
\item .\{ 端子1, 端子2, ... \} = 値 ;
\item .\{ レジスタ1, レジスタ2, ... \} := 値 ;
\end{itemize}

右辺は、整数値もしくは、左辺の連結したビット数に対応する値とします。

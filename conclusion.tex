\chapter{チュートリアルのまとめ}
\label{chap:conclusion}

NSLの言語仕様を例を交えて説明しました。50行弱で記述する小さなCPUを含め、NSLは、コンパクトに見通しよく、設計者がまさに実現したい回路を書き下ろせます。また、既存のVerilogHDLやVHDLの資産を、そのまま生かして回路を開発できるため、今までの投資を無駄にしません。

次世代のハードウェア設計言語 NSLを活用し、すばらしい設計をしてください。

\chapter{ステートマシンと状態遷移}
\label{chap:state}

論理設計において、呼び出されたときに起動する手続きではなく、静的な状態変数を持つ回路を作りたい場合があります。NSLは、このようなケースに対応し、ステートマシンを作成する構文を持ちます。状態変数は、リセット時に初期化され、状態遷移によって、遷移先の状態に書き換えます。ステートマシンは、構成要素として、モジュール、複合文の中で定義できます。複合文で定義したステートマシンは、複合文の中だけの、局所的なステートマシンとなることに注意してください。ステートマシンの定義は、下記の構文で行います。先頭の状態が、ステートマシンの初期状態になります。

\begin{reviewemlist}
\begin{alltt}
state\textunderscore{}name 状態名1, 状態名2, 状態名3, .... ;
\end{alltt}
\end{reviewemlist}

ステートマシンを定義したら、各状態に対応する動作を記述します。動作の記述は、state文を使います。

\begin{reviewemlist}
\begin{alltt}
state 状態名 実行文
\end{alltt}
\end{reviewemlist}

state文の中で、他の状態に遷移する場合、goto文を用います。シーケンスブロックのgoto文とは、全く異なることに注意してください。

\begin{reviewemlist}
\begin{alltt}
goto 状態名 ;
\end{alltt}
\end{reviewemlist}

次に、ステートマシンの例題を示します。

\reviewlistcaption{リスト12.1: tut10}
\begin{reviewlist}
\begin{alltt}
declare tut10 simulation \{\}

module tut10 \{
   state\textunderscore{}name state1, state2, state3;
   state state1 \{
       reg c1[2]=0;
       if(c1++ == 3) goto state2;
       \textunderscore{}display("in state1 \%d",c1);
   \}

   state state2 \{
       reg c2[2]=0;
       if(c2++ == 3) goto state3;
       \textunderscore{}display("in state2 \%d",c2);
   \}

   state state3 \{
       reg c3[2]=0;
       if(c3++ == 3) \{goto state1; \textunderscore{}finish();\}
       \textunderscore{}display("in state3 \%d",c3);
   \}
 \}
\end{alltt}
\end{reviewlist}


状態変数は静的であり、state文は、実行の条件がそろった上で、状態変数が指定した状態になっている場合に、実行を行うことに注意してください。たとえば、関数や手続き内にステートマシンを作成したとき、state文を記述しても、その関数もしくは手続きが動作していなければ、状態変数が一致していても、実行は行われません。

それでは、実行してみましょう。ここでは、Windows版NSL COREを用いて、シミュレーション用のVerilogHDLを作成します。LanguageがSimulationとなっていることを確認し、右下のTargetにtu10を記入した上で、tut10.nslをVerilogHDLシミュレーションファイルに変換してください。その後、Icarus Verilogでコンパイル、シミュレーションを行います。

\begin{reviewcmd}
\begin{alltt}
\textgreater{} iverilog -otut10.vvp tut10.v
\textgreater{} vvp tut10.vvp

VCD info: dumpfile tut10.vcd opened for output.
in state1 0
in state1 0
in state1 1
in state1 2
in state1 3
in state2 0
in state2 1
in state2 2
in state2 3
in state3 0
in state3 1
in state3 2
in state3 3
\end{alltt}
\end{reviewcmd}

シミュレーションで、最初に state1の0の値が2行出ているのは、リセット時にもクロックが入力されていて、(\textunderscore{}display文など)リセット信号が入っていない回路が動作しているからです。

ステートマシンの利用上、注意していただきたい点があります。状態変数の初期化はリセットのときに行われますが、通常の動作中は行われないので、手続きに記述した状態を、手続きの呼出しごとに初期状態に戻したい場合、手続きの終了前に、初期状態に遷移するgoto文を記述する必要があります。

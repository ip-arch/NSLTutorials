\chapter{浮動小数点加算器例題}
\label{chap:fpexample}

IEEE754浮動小数点形式を入力とする浮動小数点加算器の例題を示します。この加算器はパイプライン構造となっていて、1クロックごとに演算結果を出力することができます。

\reviewlistcaption{リスト15.1: fpadd}
\begin{reviewlist}
\begin{alltt}
/*
  IEEE 754 type Single Precision Floating Point Pipeline Adder
  Copyright (c) 2011 Naohiko Shimizu, IP ARCH, Inc.

  This circuit is provided only for demonstration of NSL description.

    This program is free software: you can redistribute it and/or modify
    it under the terms of the GNU General Public License as published by
    the Free Software Foundation, either version 3 of the License, or
    (at your option) any later version.

    This program is distributed in the hope that it will be useful,
    but WITHOUT ANY WARRANTY; without even the implied warranty of
    MERCHANTABILITY or FITNESS FOR A PARTICULAR PURPOSE.  See the
    GNU General Public License for more details.

    You should have received a copy of the GNU General Public License
    along with this program.  If not, see \textless{}http://www.gnu.org/licenses/\textgreater{}.


*/

/*
  Adder interface declaration.
  We will have two inputs 'a' and 'b' those must be normalized
  single precision IEEE754 numbers.
  To start the circuit, you can invoke 'exe' with two arguments.
  When the circuit finish the calculation, it will send the result
  with done signal.
  The circuit is completely pipelined.
  And you can invoke it in every cycle.
*/

declare Ieee754SpAdd \{
	input a[32];
	input b[32];
	output result[32];
	func\textunderscore{}in exe(a,b);
	func\textunderscore{}out done(result);
\}

/*
   The struct Ieee754Sp defines single precision format of IEEE754.
   It has Sign bit, 8bit Exponent, 23bit Mantissa.
*/

struct Ieee754Sp \{
	Sign;
	Exponent[8];
	Mantissa[23];
\};

/*
  We will use Barrel Shifter for pre-shift the mantissa.
  Though the required length is not 32bit, we will use 32bit shifter.
  Logic synthesizer will do the compaction of unused net.
*/

declare BarrelShift \{
	input a[32];
	input sa[8];
	output f[32];
	func\textunderscore{}in exe(a,sa):f;
\}

/*
  After the mantissa calculation, we need suppress leading zeros
  of the result.
  LeadingZeroShift provide the function.
  It will return the result and shifted amount.
*/

declare LeadingZeroShift \{
	input a[32];
	output shamt[8];
	output f[32];
	func\textunderscore{}in exe(a):f;
\}

/*
  Barrel Shifter body.
  We will decode each bit of shift amount and
  determine to shift the contents or not.
*/

module BarrelShift \{
	wire t0[32],t1[32],t2[32],t3[32],t4[32];
	func exe \{
		if(sa[0])	t0=a\textgreater{}\textgreater{}1;
		else		t0=a;
		if(sa[1])	t1=t0\textgreater{}\textgreater{}2;
		else		t1=t0;
		if(sa[2])	t2=t1\textgreater{}\textgreater{}4;
		else		t2=t1;
		if(sa[3])	t3=t2\textgreater{}\textgreater{}8;
		else		t3=t2;
		if(sa[4])	t4=t3\textgreater{}\textgreater{}16;
		else		t4=t3;
		if(sa[7:5]==0)	return t4;
		else		return 0;
	\}
\}

/*
  Leading Zero Shifter body.
  We will evaluate continuous zero from MSB to LSB.
*/

module LeadingZeroShift \{
	wire t0[32],t1[32],t2[32],t3[32],t4[32];
	wire a0,a1,a2,a3,a4;

	func exe \{
		if(a[31:16]==0)		\{ a4 = 1; t0 = a\textless{}\textless{}16; \}
		else			\{ a4 = 0; t0=a; \}
		if(t0[31:24]==0)	\{ a3 = 1; t1 = t0\textless{}\textless{}8; \}
		else			\{ a3 = 0; t1=t0; \}
		if(t1[31:28]==0)	\{ a2 = 1; t2 = t1\textless{}\textless{}4; \}
		else			\{ a2 = 0; t2=t1; \}
		if(t2[31:30]==0)	\{ a1 = 1; t3 = t2\textless{}\textless{}2; \}
		else			\{ a1 = 0; t3=t2; \}
		if(t3[31]==0)		\{ a0 = 1; t4 = t3\textless{}\textless{}1; \}
		else			\{ a0 = 0; t4=t3; \}
		shamt = 8'(\{a4,a3,a2,a1,a0\});
		return t4;
	\}

\}

/*
  Floating Adder body.
  It has 4 stages 'stageA' through 'stageD.'
  At 'stageD' we will get the result.
  exe:    invoke the circuit and calculate exponents differences.
  stageA: pre-shift for adder operation.
  stageB: Mantissa addition.
  stageC: Leading zero shift.
  stageD: Return result.
*/

module Ieee754SpAdd \{
/*
  Resources defenitions.
*/
	BarrelShift bshft;
	LeadingZeroShift lzshft;

/*
  stageA resources
*/

	reg Aexdf[8];
	Ieee754Sp reg x, y;
	proc\textunderscore{}name stageA(Aexdf, x, y);

/*
  stageB resources
*/

	reg Bm1[32], Bm2[32], Bs1, Bs2, Bexp[8];
	proc\textunderscore{}name stageB(Bm1, Bm2, Bs1, Bs2, Bexp);

	wire s1, s2, x1[32], x2[32], r1[32];
	func\textunderscore{}self madd(s1, s2, x1, x2) : r1;

/*
  stageC resources
*/

	reg Cm[32], Cs, Cexp[8];
	proc\textunderscore{}name stageC(Cm, Cs, Cexp);

/*
  stageD resources
*/

	Ieee754Sp reg z;
	proc\textunderscore{}name stageD(z);

/*
  func exe(a,b) is the starting point of this adder.
*/

	func exe \{
		wire wdiff[9];

/*
		In IEEE 754, the exponent is biased binary.
		Therefore, negative value of subtraction will
		show that 'b.Exponent \textgreater{} a.Exponent.'
		We will select pre-shift argument depending on the result.
*/
		wdiff = 9'((Ieee754Sp)(a).Exponent) - 9'((Ieee754Sp)(b).Exponent);
		if(wdiff[8]) \{
			stageA( -wdiff[7:0], b, a );
		\}
		else \{
			stageA(  wdiff[7:0], a, b );
		\}
	\}

/*
	proc\textunderscore{}name stageA(Aexdf, x, y);

	We will pre-shift 'y' for addition.
	Because IEEE754 suppress MSB's '1' in mantissa, we will add it here.
	But if the exponent part is zero, it may be un-normalized value.
	Therefore, we will not add '1'.
*/
	proc stageA \{
		wire xmsb[3],ymsb[3];

		if(x.Exponent==0)
			xmsb=0;
		else
			xmsb=3'b1;

		if(y.Exponent==0)
			ymsb=0;
		else
			ymsb=3'b1;

		stageB( \{xmsb,x.Mantissa,6'b0\},
			bshft.exe(\{xmsb,y.Mantissa,6'b0\},Aexdf),
			x.Sign,
			y.Sign,
			x.Exponent );
	\}

/*
	proc\textunderscore{}name stageB(Bm1, Bm2, Bs1, Bs2, Bexp);

	Add two mantissa values.
	In IEEE754, the mantissa does not have sign it self.
	Therefore, if the addition shows negative value,
	we will make 2's complement of it.
*/

/*
	s1,s2 is sign bit for x1,x2.
	When the value is negative, we will make 2's complement.
*/
	func madd \{
		return (32\#s1\textasciicircum{}x1) + 32'(s1) + (32\#s2\textasciicircum{}x2) + 32'(s2);
	\}

	proc stageB \{
		wire m3[32];

		m3 = madd(Bs1,Bs2,Bm1,Bm2);
		if(m3[31])
			stageC( -m3, m3[31], Bexp );
		else
			stageC(  m3, m3[31], Bexp );
	\}

/*
	proc\textunderscore{}name stageC(Cm, Cs, Cexp);

	We now have mantissa 'Cm', sign 'Cs', exponent 'Cexp.'
	But we need to suppress leading zero of mantissa for normalize.
	If the result of mantissa calculation was 0, we will return 0.
	Or if the exponent was 0, it shows un-normalized value,
	then we will not do the leading zero shift.
*/
	proc stageC \{
		if(Cm==0)
			stageD(0);
		else if(Cexp==0)
			stageD(\{Cs,Cexp,Cm[28:6]\});
		else
			stageD( \{Cs,
				8'(Cexp - lzshft.shamt + 1),
				lzshft.exe(\{Cm[30:0], 1'b0\})[30:8]
				\} );
	\}

/*
	proc\textunderscore{}name stageD(z);
	return ack signal and the result from adder.
*/
	proc stageD \{
		done(z);
		finish;
	\}
\}
\end{alltt}
\end{reviewlist}


\chapter{整数変数と一時変数および構造展開}
\label{chap:integer}

NSLの多くの構文は、生成するハードウェアと一対一に対応しています。これは、ハードウェア設計者に見通しのよい言語を提供するためです。しかし、似たような構造を複数作成する場合には、記述量の圧縮のため、構造を展開する構文が便利に使えます。 NSLは、構造展開のために、整数変数、一時変数および構造展開構文を提供します。これらの構文は、コンパイル時に評価し、値が確定します。評価は、NSL文の記述順に行われ、実行時の動作順序とは無関係となることに注意してください。

整数変数(integer)は、整数を記述できる部分に使えます。また、整数変数同士および整数の演算式を利用できます。

\begin{reviewemlist}
\begin{alltt}
integer 変数名 ;
\end{alltt}
\end{reviewemlist}

特別なケースとして、if文の条件に整数もしくは整数変数からなる式を指定した場合、コンパイル時に、真もしくは偽のいずれかの実行文のみ生成します。プリプロセッサによる条件コンパイルよりも、細やかな条件制御を可能としますが、多用すると、可読性の低下をもたらすことになるので、注意してください。

一時変数(variable)は、幅を持ち、端子と同様に用います。ただし、端子と異なり、同一クロックで複数回の転送が可能であり、かつ、部分的な転送も行えます。これは、一時変数は、転送が出現するたびに、新たな端子を自動作成し、必要な演算を行うからです。一時変数の評価は記述順に行い、展開後の端子への転送は並列に実行します。

\begin{reviewemlist}
\begin{alltt}
variable 変数名[ビット幅] ;
\end{alltt}
\end{reviewemlist}

整数変数と一時変数を用いて回路を生成するための、generate文があります。

\begin{reviewemlist}
\begin{alltt}
generate(整数変数=初期値 ; 整数変数式 ; 整数変数更新式 ) 実行文
\end{alltt}
\end{reviewemlist}

次の例題は、generate構文を使い、巡回符号による擬似乱数生成回路を作成しています。一時変数vの各ビットへの値をgenerate文を用いて設定し、生成した結果をまとめてレジスタrに転送します。

\reviewlistcaption{リスト13.1: tut11}
\begin{reviewlist}
\begin{alltt}
declare glfsr \{
   input seed[16];
   output q[16];
   func\textunderscore{}in set\textunderscore{}seed(seed);
   func\textunderscore{}in next\textunderscore{}rand : q;
\}

module glfsr \{
   reg r[16] = 0x39a5;
   variable v[16];
   integer i;
   func next\textunderscore{}rand \{
       generate (i=0;i\textless{}15;i++) \{
           if((i == 13) \textbar{}\textbar{} (i == 12) \textbar{}\textbar{} (i == 10)) v[i] = r[i+1] \textasciicircum{} r[0];
           else v[i] = r[i+1];
           \}
       v[15] = r[0];
       r:=v;
       return r;
   \}

   func set\textunderscore{}seed r:=seed;
\}

declare tut11 simulation \{\}

module tut11 \{
   glfsr rmod;
   reg count[16]=0;
   count++;
   any \{
       3'(count) == 7: \{
           \textunderscore{}display("set seed:\%d",count+0x9876);
           rmod.set\textunderscore{}seed(count+0x9876);
       \}
       count==10:    \textunderscore{}finish("finished");
       else:    \textunderscore{}display("random generate \%d", rmod.next\textunderscore{}rand());
   \}
\}
\end{alltt}
\end{reviewlist}


この例題を実行すると、下記のようになります。ここでは、Windows版NSL COREを用いて、シミュレーション用のVerilogHDLを作成します。LanguageがSimulationとなっていることを確認し、右下のTargetにtu11を記入した上で、tut11.nslをVerilogHDLシミュレーションファイルに変換してください。その後、Icarus Verilogでコンパイル、シミュレーションを行います。

\begin{reviewcmd}
\begin{alltt}
\textgreater{} iverilog -otut11.vvp tut11.v
\textgreater{} vvp tut11.vvp

VCD info: dumpfile tut11.vcd opened for output.
random generate 14757
random generate 14757
random generate 43218
random generate 21609
random generate 40500
random generate 20250
random generate 10125
random generate 42950
set seed:39037
random generate 39037
random generate 63550
finished
\end{alltt}
\end{reviewcmd}

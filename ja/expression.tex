\chapter{数値表現と整数表現}
\label{chap:expression}

NSLでは、演算に用いる数値表現にはビット幅を明示します。これは、ハードウェアの演算では、数値のビット幅が回路規模に大きな影響を与えるからです。NSLの式やレジスタ、端子の値は符号なし2進数として扱います。設計者にとって自明なビット幅となる演算を行う場合には、ビット幅を明示する記述は冗長なので、NSLでは、明らかにビット幅が推定可能な部分においては、整数を指定することができます。自明な場合とは、レジスタや端子への値の転送、同一ビット幅同士の2項演算です。シフト演算におけるシフト数の指定は整数での表記を主とします(整数以外を与えた場合、バレルシフタが生成されるので、コンパイラはウォーニングを出します)。

ビット数を明示する数値表現には次の7種類があります。

\begin{enumerate}
\item 2進数: 0bから始まる2進数は、表記したビット数を有する数値として扱います。例:0b100
\item 8進数: 0oから始まる8進数は、表記した桁×3ビットの数値として扱います。例:0o13
\item 16進数: 0xから始まる16進数は、表記した桁×4ビットの数値として扱います。例:0x123
\item 2進数: 数値'b から始まる2進数は、数値で示すビット幅を持つ数値として扱います。例: 4'b1
\item 8進数: 数値'o から始まる10進数は、数値で示すビット幅を持つ数値として扱います。例:8'o25
\item 10進数: 数値'd から始まる10進数は、数値で示すビット幅を持つ数値として扱います。例:8'd20
\item 16進数: 数値'h から始まる16進数は、数値で示すビット幅を持つ数値として扱います。例:8'h3
\end{enumerate}

NSLでは、整数は、32ビット符号付の範囲を持ちます。レジスタやメモリの初期値や、ビット幅を推定可能な部分においても、32ビット符号付整数で表せる数よりも大きな数値を必要とする場合には、整数による初期化ではなく、明示的にビット数を指定した数値表現を使います。

注:式や端子の値は符号なしに対して、整数は符号付なので、両者を混在して計算を行う場合、十分注意してください。

\section{浮動小数点数}

信号処理などで用いる数式の係数計算など、回路機能の記述を容易にするために浮動小数点数を使うことができます。浮動小数点数は、論理合成対象ではありませんが、コンパイル時に評価し、値を端子やレジスタ、構造体に定数として転送することができます。また、\textunderscore{}int関数で整数に変換すれば、値を整数定数として演算式において利用可能です。

浮動小数点数は、小数点を含む数値列、および、浮動小数点数E指数の形で表記します。浮動小数点数の演算は、加減乗除のほかに、次の初等関数が利用可能です。

\textunderscore{}real, \textunderscore{}int, \textunderscore{}acos, \textunderscore{}asin, \textunderscore{}atan, \textunderscore{}atan2, \textunderscore{}ceil, \textunderscore{}cos, \textunderscore{}cosh, \textunderscore{}exp, \textunderscore{}fabs,
\textunderscore{}floor, \textunderscore{}fmod, \textunderscore{}log, \textunderscore{}log10, \textunderscore{}pow, \textunderscore{}sin, \textunderscore{}sinh, \textunderscore{}sqrt, \textunderscore{}tan, \textunderscore{}tanh

式は、整数との混在はできません。\textunderscore{}realを用いて浮動小数点数に変換するか、浮動小数点数だけを用いてください。

回路に計算結果を転送する場合、値を整数値として扱う場合には、\textunderscore{}int(浮動小数点式)として転送します。IEEE754形式のビット列として転送する場合には、式を転送の右辺に記述します。転送先は、32ビットまたは64ビットの、wire,regもしくは、構造体でなければなりません。この転送において、浮動小数点数は、IEEE754浮動小数点単精度もしくは倍精度のビット列に変換されます。メモリ、レジスタなどの初期値としても利用できます。

例  Z = \textunderscore{}sin(45./180.);    /* IEEE754形式ビット列として転送 */
Y = \textunderscore{}int(\textunderscore{}sin(45./180.)*127.)+128; /* 計算結果の整数を転送 */

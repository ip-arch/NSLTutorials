\chapter{プリプロセッサ}
\label{chap:preprocessor}

NSLは、プリプロセッサディレクティブとして、次のものを用意しました。

\begin{itemize}
\item マクロ定義

\begin{itemize}
\item \#define マクロ名 定義値: マクロを定義します。ソースコード中に書かれたマクロ名は、定義した値に変換します。識別子中に、マクロ名を利用する場合、\%\%で囲みます。
\item \#undef マクロ名:  マクロ定義を取り消します。
\end{itemize}

\item 条件コンパイル

\begin{itemize}
\item \#ifdef マクロ名: マクロが定義されている場合、以降\#endifもしくは\#elseまでのソースコードをコンパイルします。
\item \#ifndef マクロ名: マクロが定義されていない場合、以降\#endifもしくは\#elseまでのソースコードをコンパイルします。
\item \#if 数値 : 数値が0以外の場合、以降\#endifもしくは\#elseまでのソースコードをコンパイルします。
\item \#else : \#ifdef/\#ifndefの逆の条件で、以降\#endifまでのソースコードをコンパイルします。
\item \#endif : 条件コンパイルの停止
\end{itemize}

\item インクルード

\begin{itemize}
\item \#include \textless{}ファイル名\textgreater{} : NSL\textunderscore{}INCLUDE環境変数のパスから、ファイル名を検索し、ソースコードとして挿入します。
\item \#include "ファイル名" : カレントディレクトリもしくは、コンパイル時に指定したパスから、ファイル名を検索し、ソースコードとして挿入します。
\end{itemize}

\end{itemize}

プリプロセッサである、nslpp.exeの出力は、C言語のプリプロセッサと互換性があります。そこで、さらに高度なマクロを利用したい場合、gccなどをNSLのフロントエンドとして使うこともできます。
